% Template file for an a0 portrait poster.
% Written by Graeme, 2001-03 based on his SOC poster.
%
% See discussion and documentation at
% <http://www.astro.gla.ac.uk/users/norman/docs/posters/>
%
% $Id: poster-template-portrait.tex,v 1.2 2002/12/03 11:25:55 norman Exp $



% We switch to portrait mode. This works as advertised.
\documentclass[a0,landscape]{a0poster}
% You might find the 'draft' option to a0 poster useful if you havei
% lots of graphics, because they can take some time to process and
% display. (\documentclass[a0,draft]{a0poster})

% Switch off page numbers on a poster, obviously, and section numbers too.
\pagestyle{empty}
\setcounter{secnumdepth}{0}

% The textpos package is necessary to position textblocks at arbitary
% places on the page.
\usepackage[absolute]{textpos}

% Graphics to include graphics. Times is nice on posters, but you
% might want to switch it off and go for CMR fonts.
\usepackage{graphics,graphicx,wrapfig,times, subfig}

\usepackage[T1]{fontenc}  % French
\usepackage{harvard}
\usepackage{amsmath,amsfonts,amssymb}
%\usepackage{gensymb}  %  for degree symbol
\usepackage{fancybox}
% These colours are tried and tested for titles and headers. Don't
% over use color!
\usepackage{framed, color}
\definecolor{DarkBlue}{rgb}{0.1,0.1,0.5}
\definecolor{Red}{rgb}{0.9,0.0,0.1}
\definecolor{vert}{rgb}{0.1,0.7,0.2}
\definecolor{brique}{rgb}{0.7,0.16,0.16}
\definecolor{blue}{rgb}{0.36, 0.54, 0.66}  % airforceblue
\definecolor{navy}{rgb}{0.0, 0.0, 0.5}        % navyblue

\definecolor{MonStabilo}{rgb}{0.1,0.1,0.5}   % m�me def que darkblue, mais pour highlight de mots cl�s
\definecolor{asparagus}{rgb}{0.53, 0.66, 0.42}


% see documentation for a0poster class for the size options here
\let\Textsize\normalsize
\def\Head#1{\noindent\hbox to \hsize{\hfil{\LARGE\color{DarkBlue} #1}}\bigskip}
\def\LHead#1{\noindent{\LARGE\color{DarkBlue} #1}\smallskip}
\def\Subhead#1{\noindent{\large\color{DarkBlue} #1}}
\def\Title#1{\noindent{\VeryHuge\color{brique} #1}}

% Set up the grid
%
% Note that [40mm,40mm] is the margin round the edge of the page --
% it is _not_ the grid size. That is always defined as
% PAGE_WIDTH/HGRID and PAGE_HEIGHT/VGRID. In this case we use
% 15 x 25. This gives us a wide central column for text (7 grid
% spacings) and two narrow columns (3 each) at each side for
% pictures, separated by 1 grid spacing.
%
% Note however that texblocks can be positioned fractionally as well,
% so really any convenient grid size can be used.
%
\TPGrid[40mm,40mm]{15}{25}  % 3 - 1 - 7 - 1 - 3 Columns

% Mess with these as you like
\parindent=0pt
%\parindent=1cm
\parskip=0.5\baselineskip

% abbreviations
\newcommand{\ddd}{\,\mathrm{d}}

\begin{document}

% Understanding textblocks is the key to being able to do a poster in
% LaTeX. In
%
%    \begin{textblock}{wid}(x,y)
%    ...
%    \end{textblock}
%
% the first argument gives the block width in units of the grid
% cells specified above in \TPGrid; the second gives the (x,y)
% position on the grid, with the y axis pointing down.

% You will have to do a lot of previewing to get everything in the
% right place.

% This gives good title positioning for a portrait poster.
% Watch out for hyphenation in titles - LaTeX will do it
% but it looks awful.

\begin{textblock}{15}(0,0)
\baselineskip=3\baselineskip
\begin{center}
\Title{The effects of extreme climatic events on dairy farmers' risk preferences:\\
A nonparametric approach}
\end{center}
\end{textblock}

\begin{textblock}{15}(0,2.5)
\begin{center}
\LHead{\textbf{Christophe Bontemps} \& St�phane Couture}\\
\Large{\textit{Toulouse School of Economics (INRA) \& INRA-MIAT-Toulouse}}
\end{center}
\end{textblock}

% Put the GU logo in the top right.i
\begin{textblock}{5}(2.5,1.4)
%\includegraphics[height=50mm]{Graphics/LogoTSE.jpg}
\includegraphics[height=90mm]{Graphics/TSE2.pdf}
\end{textblock}


\begin{textblock}{1.5}(0,2)
{\color{asparagus}
\textit{This work was funded through the project FARMATCH within 
the Metaprogramme INRA-ACCAF}}
\end{textblock}


\begin{textblock}{5}(12,1.8)
\includegraphics[height=40mm]{Graphics/MIA_logo.png} \hspace{1cm} \includegraphics[height=40mm]{Graphics/Logotype-INRA_medium.jpg}

\end{textblock}

% First column
\begin{textblock}{4.5}(0,4.5)

 \LHead{Motivation}

Climate change is likely to increase average daily temperatures and the frequency of heat waves, which can reduce meat and milk production (Key and Sneeringer 2014)

\begin{figure}[!h]
\centering
\includegraphics[height=80mm]{Graphics/TempOverTime.pdf}
\caption{Average temperature (C) between 1996 and 2006 in the South-West of France}
\end{figure}

Managing the risk of such intense events may influence dairy farmers' production decisions, and their risk preferences.
The idea is to study precisely how a realized extreme event affects farmers' risk preferences.

\begin{itemize}
  \item[] {\color{DarkBlue} \textbf{Research questions:}}
  \begin{itemize}
    \item[{\color{DarkBlue} $\bullet$}] {\color{DarkBlue} \textit{Is there a change observed in dairy farmers' risk preferences over time?}}
    \item[{\color{DarkBlue} $\bullet$}]  {\color{DarkBlue} \textit{Do extreme climatic events modify dairy farmers' risk aversion?}}
    \item[{\color{DarkBlue} $\bullet$}]  {\color{DarkBlue} \textit{Is a nonparametric approach adapted/tractable to answer these question ?}}
  \end{itemize}
\end{itemize}


\LHead{Analytical framework}

 The usual way of investigating the production risk into a {\color{MonStabilo}stochastic production function} is to consider a Just and Pope \citeyear{Just}, \citeyear{Just2} production function given by:
\begin{equation} \label{JustPope}
y=f(x,z)+g(x,z)\epsilon
\end{equation}
where $y$ is the observed output quantity, $x$ is a vector of variable input quantities $(x_1, . . . , x_J)$,
$z$ is a vector of quasi-fixed input quantities $(z_1, . . . , z_K)$, $f(.)$ is the mean production
function, $g(.)$ is the production risk function. The random term $\epsilon$ represents a {\color{MonStabilo}weather shock} that may affect output, exogenous to farmer's action, with zero mean and a variance of one.

The dairy farmer's optimisation programme is written as follows:
\begin{equation*}
Max_{x} \; \;  EU(\pi) = EU\bigg(pf(x,z)+pg(x,z) \epsilon -cx\bigg)
\end{equation*}
where $p$ denotes the milk production price, $c$ the vector of variable input prices.\\


We get the the following first-order conditions (FOC):
\begin{equation*}
E\bigg[U'(\pi)\big(pf_j(x,z)+pg_j(x,z) \epsilon -c_j\big)\bigg]=0 \; \; \;  \forall j=1,..., J
\end{equation*}

where $U'(.)$ is the marginal utility of profit, $f_j$ and $g_j$ denote the first derivatives of the mean production function and the risk production function, respectively, with respect to the $j$-th variable input. Rewritten in the following way:

\begin{equation}\label{RiskTheo}
 pf_j(x,z) - c_j - \theta(.) pg_j(x,z)=0 \; \; \; \forall j=1,..., J
\end{equation}

where $\theta(x,z,p,c) =  \frac{E[U'(\pi)\epsilon]}{E[U'(\pi)]}$ is the {\color{MonStabilo}risk preference function}.

Using a first-order polynomial approximation (see Kumbhakar and Tsionas \citeyear{KumbhakarTsionas}) the risk preference function $\theta(.)$ takes the following form:
\begin{equation}
\theta(.)=-AR(\pi)\sigma_{\pi}
\end{equation}

where $AR(\pi)=\frac{-U''(\pi)}{U'(\pi)}$ is the {\color{MonStabilo}Arrow-Pratt measure of absolute risk aversion} and $\sigma_{\pi}^2=var(\pi)=p^2(g(x,z))^2$.

\end{textblock}


% Seconde colonne
\begin{textblock}{4.5}(5.5,4.5)

\LHead{Nonparametric estimation}

We follow the multi-step procedure proposed by Kumbhakar and Tsionas \citeyear{KumbhakarTsionas} for estimating  the mean production function $f(\cdot)$ , and the production risk function $g(\cdot)$ leading to the risk preference function $\theta(\cdot)$.
\begin{itemize}
\item In a first stage, we estimate mean production function $f(\cdot)$ :
\begin{eqnarray*}
y & = & f(x,z)+g(x,z)\epsilon \\
  & = & f(w)+\nu
\end{eqnarray*}
where $w$ denotes the vector of all variable inputs (including variable inputs and quasi-fixed inputs), and $\nu$ is the error term. The function $f()$ can then be estimated by $\widehat{f()}$ using classical nonparametric regression methods.
\begin{equation*}
 \widehat{f}(w) =  \frac{\sum_{i=1}^n Y_i \; K\left( \frac{W_i -w}{h}\right) }
    {\sum_{i=1}^n K\left( \frac{W_i - w}{h}\right)}
\end{equation*}
Where $K()$ is a multivariate kernel function and $h$ is a vector of bandwidths associated to the set of explanatory variables $w$. Since we are interested mainly by the derivatives of $f()$, we use the {\color{MonStabilo}local linear nonparametric estimation }procedure proposed by Li and Racine \citeyear{Li2004} allowing simultaneous estimation of both the function and its derivatives $f_j(w)$ for $j=1, \ldots, J$.


\item In the second stage, we compute the sample residuals $\widehat{e_i} = Y_i - \widehat{f}(W_i)$ of the first stage regression model.

Then we use a local linear nonparametric estimator of $e_i$ (resp. $e_i^2)$ on $w$ to compute the estimator of the  mean risk production function  $\widehat{g}(w)$  and its derivatives $\widehat{g _j}(w)$ for $j=1, \ldots, J$ (resp. the variance $\widehat{\sigma^2}(w)$).

\item Once the mean production function and the mean risk production function and their derivatives have been estimated, we compute the  risk preference function $\theta(\cdot)$ using the FOC in equation \ref{RiskTheo}.

\begin{equation}\label{thetahat}
\hat{\theta}(\cdot) = \frac{1}{J} \sum_{j=1}^{J} \Bigl[ \frac{{\hat f_j(X)}-c_j/p}{-{\hat g_j(X)}}\Bigr]
\end{equation}

\end{itemize}

\LHead{Empirical application}

\LHead{{\large Data}}

The sample consists of 2588 dairy farmers from six regions in the Southwestern of France. The period covered is from 1996-2006. Thus total number of observations is 28458. The farm-level data were complemented with weather data for each region from the French Meteorogical Institute.\\

\begin{table}[!h]
\centering
\begin{tabular}{llrrrr}\hline\hline
 & \textbf{Variable} & $\mathbf{mean}$ & $\mathbf{sd}$ & \textbf{min} & \textbf{max} \\
  \hline
$ Y$ &  Milk. Prod.(1000 L) &  251.78 & 123.34 & 17.01 & 1407.11 \\
$X_1$& Irrigated Land (ha) &    4.30 &   7.53 &  0.00 &   80.00 \\
$X_2$& Purchased feed (kg/cow) & 1420.60 & 424.65 &  0.00 & 8294.00 \\
$X_3$& Farm Land (ha) &   65.12 &  42.42 &  0.10 &  997.00 \\
$X_4$& Forage crop (ha) &   42.62 &  26.87 &  0.00 &  300.10 \\
$X_5$& Livestock (Heads) &   64.04 &  29.77 &  6.90 &  367.40 \\
$Z_1$& Milk Quota (1000 L) &  214.65 & 121.69 &  0.00 & 2102.74 \\
$Z_2$& Temp (C) &   13.23 &   0.85 & 10.26 &   14.76 \\
$Z_3$& Evapotranspiration &    2.56 &   0.27 &  2.09 &    3.32 \\
$Z_4$& Hydric Stress  &   -0.32 &   0.65 & -1.70 &    0.91 \\
  \hline\hline
  \end{tabular}

\caption{Descriptive statistics, (1996-2006)}
\label{StatYear}
\end{table}

\end{textblock}


% troiseimen  colonne
\begin{textblock}{4.5}(10.5,4.5)

\LHead{{\large Nonparametric estimation implementation}}


We use up-to-date nonparametric estimation techniques to compute the ingredients needed for estimating $\theta (\cdot)$ according to equation \ref{thetahat}. As in any nonparametric estimation, the choice of the bandwidth a is a crucial element in the practical implementation. For both the computation of $\widehat{f}(\cdot)$, $\widehat{g}(\cdot)$ and $\widehat{\sigma^2}(\cdot)$, we opted for the computation of {\color{MonStabilo}cross-validated bandwidths} for each each year so that the local linear estimators are automatically balanced between bias and variance. We choose higher order continuous kernels implemented in the R package \textit{np} (Hayfield and Racine \citeyear{Hayfield2008}) %\\

We use another interesting feature of the recent development in nonparametric estimation technique by using {\color{MonStabilo} Kernel Regression Significance Tests}. We run this test based on the work by Racine, Hart, and Li \citeyear{Racine2006} for each year and derive significance of each explanatory variable (399 bootstraps). Hence,  we confirm the significance observed in running a linear regression (t-test). %\\

Finally, we also check \textit{ex-post} whether the risk production function estimated where satisfying classical production function features ($ f' > 0$ and $f'' <0$).
%One possible extension would be to estimate $f(\cdot)$ under shape constrains (see Du, Parmeter and Racine \citeyear{Du2013})


\LHead{{\large Results }}

As an illustration, we report  the partial nonparametric regression plots and results of the significance test for the production function $\widehat{f}(\cdot)$ for the year 2003.

\begin{tabular}{ccccc}
 &\includegraphics[height=50mm]{Graphics/Poster-Fplot-haspirri.pdf} &
\includegraphics[height=50mm]{Graphics/Poster-Fplot-concvlr.pdf} &
\includegraphics[height=50mm]{Graphics/Poster-Fplot-sau.pdf} &
\includegraphics[height=50mm]{Graphics/Poster-Fplot-cows.pdf} \\
& \multicolumn{4}{c}{}  \\
& for \textit{Irrigated Land} &  for \textit{Feed} & for \textit{Farm land} & for \textit{Heads}  \\
%\multicolumn{2}{l}{Kernel Significance test :} & & \\
P-value :  &  $<2e-16^{***}$  &  $<2e-16^{***}$ &  $<2e-16^{***}$& $ 2e-16^{***}$
 \end{tabular}


We provide below very preliminary results of the {\color{MonStabilo} distribution of the AR} nonparametrically estimated for each dairy farm each year with a special emphasis on the extreme climatic event year 2003 (in red).\\


\begin{figure}[!h]
\centering
\includegraphics[height=90mm, width=200mm]{Graphics/Poster-PlotAR.pdf}
\caption{Distribution of estimated AR over time (1996-2006) }
\end{figure}

To {\color{MonStabilo} conclude}, the distribution of the nonparametric estimation of dairy farmer's risk aversion is significantly higher when an extreme climatic event occurs.

%\end{textblock}
%
%\begin{textblock}{4.5}(10.5,17.5)
%  \LHead{Bibliography}\\

{\small
\bibliographystyle{agsm}
\bibliography{../AGRO_RISK3}
}
\end{textblock}
\end{document}

